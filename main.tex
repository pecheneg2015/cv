\PassOptionsToPackage{dvipsnames}{xcolor}

\documentclass[10pt,a4paper,ragged2e,withhyper]{altacv}

\geometry{left=1.2cm,right=1.2cm,top=1cm,bottom=1cm,columnsep=0.75cm}

\usepackage{paracol}
\usepackage[T2A]{fontenc}     % внутренняя T2A кодировка TeX
\usepackage[koi8-r]{inputenc} % кодировка - можно использовать [cp866] [cp1251]
\usepackage[russian]{babel}   % включение переносов

\ifxetexorluatex
  \setmainfont{Roboto Slab}
  \setsansfont{Lato}
  \renewcommand{\familydefault}{\sfdefault}
\else
  \usepackage[rm]{roboto}
  \usepackage[defaultsans]{lato}
  \renewcommand{\familydefault}{\sfdefault}
\fi

\ifdarkmode%
  \definecolor{PrimaryColor}{HTML}{C69749}
  \definecolor{SecondaryColor}{HTML}{D49B54}
  \definecolor{ThirdColor}{HTML}{1877E8}
  \definecolor{BodyColor}{HTML}{ABABAB}
  \definecolor{EmphasisColor}{HTML}{ABABAB}
  \definecolor{BackgroundColor}{HTML}{191919}
\else%
  \definecolor{PrimaryColor}{HTML}{001F5A}
  \definecolor{SecondaryColor}{HTML}{0039AC}
  \definecolor{ThirdColor}{HTML}{F3890B}
  \definecolor{BodyColor}{HTML}{666666}
  \definecolor{EmphasisColor}{HTML}{2E2E2E}
  \definecolor{BackgroundColor}{HTML}{E2E2E2}
\fi%

\colorlet{name}{PrimaryColor}
\colorlet{tagline}{SecondaryColor}
\colorlet{heading}{PrimaryColor}
\colorlet{headingrule}{ThirdColor}
\colorlet{subheading}{SecondaryColor}
\colorlet{accent}{SecondaryColor}
\colorlet{emphasis}{EmphasisColor}
\colorlet{body}{BodyColor}
\pagecolor{BackgroundColor}
\NewInfoField{telegram}{\faTelegram}[https://t.me/]

\renewcommand{\namefont}{\Huge\rmfamily\bfseries}
\renewcommand{\personalinfofont}{\small\bfseries}
\renewcommand{\cvsectionfont}{\LARGE\rmfamily\bfseries}
\renewcommand{\cvsubsectionfont}{\large\bfseries}

\renewcommand{\itemmarker}{{\small\textbullet}}
\renewcommand{\ratingmarker}{\faCircle}

\begin{document}
    \name{Печенин Дмитрий}
    \tagline{SENIOR FRONTEND DEVELOPER}
    \photoL{4cm}{photo}

    \personalinfo{
        \email{d@pechen.in}\smallskip
        \phone{+7(965)036-63-41}
        \location{Санкт-Петербург, Россия}\\
        \linkedin{dpechenin}
        \telegram{pechen_in}
    }

    \makecvheader
    \columnratio{0.35}
    \begin{paracol}{2}
        \cvsection{СПЕЦИАЛИЗАЦИЯ}
            \begin{quote}
                Разработка высокопроизводительных  web приложений с повышенными требованиями к надёжности и отзывчивости.
            \end{quote}
        \cvsection{НАВЫКИ}
            \cvtag{Typescript}
            \cvtag{Javascript}
            \cvtag{CSS}
            \cvtag{SCSS}
            \cvtag{HTML5}
            \cvtag{Webpack}
            \cvtag{Vite}
            \cvtag{Elm}
            \cvtag{SQL}
            \cvtag{Docker}
            \cvtag{Java}
            \cvtag{microfrontend}
            \cvtag{Webpack Module Federation}
            \cvtag{FP}


            \medskip
            \cvtag{ReactJS}
            \cvtag{React Table}
            \cvtag{axios}
            \cvtag{imask}
            \cvtag{NextJS}
            \cvtag{Formik}
            \cvtag{React Router}
            \cvtag{Redux}
            \cvtag{Effector}
            \cvtag{MobX}
            \cvtag{NestJS}
            \cvtag{NGINX}
            \cvtag{REST}
            \cvtag{GraphQL}
            \cvtag{SCORM}


             \medskip
             \cvtag{Windows}
             \cvtag{Linux}
             \cvtag{bash}
             \cvtag{cmd}
             \cvtag{PowerShell}
             \cvtag{git}
            \cvtag{websockets}


             \medskip
             \cvtag{Jest}
             \cvtag{Cypress}
             \cvtag{Playwright}


        \cvsection{Личные качества}
            \cvtag{целеустремлённость}
            \cvtag{креативность}
            \cvtag{ответственность}
            \cvtag{пунктуальность}
            \cvtag{аналитическое мышление}
            \cvtag{гибкость}
            \cvtag{обучаемость}
            \cvtag{любознательность}
            \cvtag{умение работать в команде}
            \cvtag{терпение}
            \cvtag{стрессоустойчивость}
            \newpage


        \cvsection{Знание языков}
            \cvlang{Русския язык}{родной}\\
            \medskip
            \cvlang{Английский язык}{B1}


        \cvsection{Увлечения}
            \cvtag{Чтение}
            \cvtag{Настольные игры}
            \cvtag{Кино}
            \cvtag{Путешествия}
            \cvtag{Функциональные ЯП}
            \cvtag{Спорт}

        \newpage
        \switchcolumn

        \cvsection{О себе}
            \begin{quote}
                Меня зовут Дмитрий. Я  senior frontend  разработчик из города Санкт-Петербурга.
                Мне довелось поработать как в стартапах и  web-студиях, так и в крупных компаниях.
            \end{quote}
            \begin{quote}
                Самым интересным и сложным проектом, над котором мне довелось работать, была B2B  платформа для букмекерского бизнеса, т.к. она имела высокие требования к скорости работы.
                Данный проект был аккредетован Министерством по налогам и сборам РБ.
            \end{quote}
            \begin{quote}
                Также я принимал участие в разработке банковских приложений, edtech  платформы (B2B  и B2C), разработке решения для работы с webpack module federation и ряде других интересных проектов.
            \end{quote}
            \begin{quote}
                Основой успешного проекта считаю общение - общение внутри команды, общение "бизнеса"{ }и  "разработки". Также, не менее важным считаю умение находить компромиссы и оптимальные решения.
            \end{quote}
        \cvsection{Опыт работы}
            \cvevent{Ведущий эксперт по технологиям }{| СберОбразование}{сентябрь 2021 -- по н.в}{СПб,РФ}
            \textbf{Обязанности}
            \begin{itemize}
                \item Разработка нового функционала
                \item Ведение библиотеки компонентов в Storybook
                \item Написание тестов (e2e и unit)
                \item Участие в разработке аналитики для нового функционала
                \item Написание документации
                \item Участие в процессе найма и адаптации новых сотрудников
                \item Участие в релизном процессе
                \item Разработка внутренних библиотек
                \item Исследовательская работа
                \item Участие в глобальных межкомандных процессах
            \end{itemize}
            \textbf{Технологический стек}: Typescript, React, GQTY, Effector, Patronum, Webpack module federation,Webpack, TailwindCSS, SCORM, SCSS.


            \divider
            \cvevent{Frontend разработчик }{| АБ Россия }{июнь 2020 -- сентябрь 2021}{СПб, РФ}
            \textbf{Обязанности}
            \begin{itemize}
                \item Разработка нового функционала
                \item Тестирование (критически важные сценарии)
                \item Совместная с дизайнерами разработка компонентов ui-kit
                \item Участие в работе над дизайн системой
                \item Участие в проектировании API
            \end{itemize}
            \textbf{Технологический стек}: Typescript, React, MobX, Webpack, Axios, Styled components, КриптоПРО, РуТокен.


            \divider
            \cvevent{Frontend разработчик }{| Premium Betting Technologies  }{декабрь 2018 -- июнь 2020}{СПб, РФ}
            \textbf{Обязанности}
            \begin{itemize}
                \item Реализация нового функционала
                \item Рефакторинг и поддержка legacy  на JSP
                \item Разработка  дополняющих основной продукт решений
                \item Участие в проектировании API
                \item Анализ работы приложения
                \item Оптимизация работы приложения
                \item Интеграция с платёжными система и сторонними сервисами
            \end{itemize}
            \textbf{Технологический стек}: Javascript, React, Redux, Webpack, Axios, Ant Design, StompJS, SockJS.


            \divider
            \cvevent{Frontend разработчик }{| НПК ИнТех  }{июль 2018 -- декабрь 2018}{СПб, РФ}
            \textbf{Обязанности}
            \begin{itemize}
                \item Разработка нового функционала
                \item Интеграция программного продукта
            \end{itemize}
            \textbf{Технологический стек}: Javascript, HighchartsJS, InfluxDB, D3.js, Leaflet.


            \divider
            \cvevent{ Дежурный инженер }{| SpaceWeb  }{июль 2017 -- июль 2018}{СПб, РФ}
            \textbf{Обязанности}
            \begin{itemize}
                \item Доработка web приложений на различных CMS: Wordpress, Joomla, OpenCart, UMI.CMS.
                \item Анализ работы почтовой системы
            \end{itemize}


            \divider
            \cvevent{Web разработчик }{| IT Media Group  }{март 2017 -- июль 2018}{СПб, РФ}
            \textbf{Обязанности}
            \begin{itemize}
                 \item Разработка web приложений на различных CMS: Wordpress, Drupal, OpenCart, Joomla.
                 \item Доработка web-приложений: реализация адаптивности, доработка функционала, разработка новых расширений для  CMS
            \end{itemize}

        \cvsection{Образование}
            \cvevent{Магистратура }{| Санкт-Петербургский государственный университет аэрокосмического приборостроения}{сентябрь 2016 -- июнь 2018}{СПб, РФ}
            \begin{itemize}
                \item 09.04.01: Информатика и вычислительная техника (математическое моделирование)
            \end{itemize}
            \divider

            \cvevent{Бакалавриат }{| Санкт-Петербургский государственный университет аэрокосмического приборостроения}{сентябрь 2012 -- июнь 2016}{СПб, РФ}
            \begin{itemize}
                \item 09.03.01: Информатика и вычислительная техника
            \end{itemize}
    \end{paracol}
\end{document}